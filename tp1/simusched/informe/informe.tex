%\documentclass[spanish,a4paper]{article}
\documentclass[11pt]{article}
\usepackage[utf8]{inputenc}
\usepackage[a4paper,margin=6em]{geometry}
\usepackage[spanish]{babel}

% Paquetes generales
\usepackage{amsmath}
\usepackage[utf8]{inputenc}%esto permite meter tildes sin el coso
\usepackage[spanish]{babel}
\usepackage{ifthen}
\usepackage{amssymb}
\usepackage{amsmath}
\usepackage{multicol}
\usepackage[absolute]{textpos}
\usepackage{hyperref}
\usepackage{enumitem}
%\usepackage{graphicx}
\usepackage{caratula}
\usepackage{float}%este es el que acomoda bien las figures

\newcommand{\linea}{\noindent\rule{12cm}{0.4pt}}

%\include{caratula}
\begin{document}


\titulo{Trabajo Pr\'{a}ctico: Scheduling}
\subtitulo{}

\fecha{\today}

\materia{Sistemas Operativos}
\grupo{}

\integrante{Arribas, Joaquín}{702/13}{joacoarribas@hotmail.com}
\integrante{Lebrero, Ignacio}{702/13}{nachitou@hotmail.com}
\integrante{Vázquez, Jésica}{702/13}{jesis\_93@hotmail.com}

\maketitle

\thispagestyle{empty}
\vspace{3cm}
\tableofcontents
\newpage
\vfill

\begin{abstract}
  En este trabajo se implementaron distintas simulaciones interactivas entre tareas. A su vez se implementaron distintas clases de scheduling para interactuar 
  con las tareas creadas, y dichas interacciones se representaron de manera gráfica.

  Hare que no se hablar.
\end{abstract}

\newpage

\section{Ejercicio 1}

El ejercicio consiste en implementar una tarea llamada \textbf{TaskConsola}, que simule una tarea que realiza llamadas bloqueantes. La tarea 
recibe por parámetro la cantidad de llamadas bloqueantes que debe realizar, y un intervalo que determina un máximo y un mínimo para la duración 
de cada una. Dicha duración es generada de manera pseudoaleatoria.

Para resolver el ejercicio creamos una función llamada \textbf{generate} que se encarga de realizar la simulación de la tarea. Genera una $semilla$ utilizando la función 
\textbf{time} y luego, para cada llamada bloqueante, genera el tiempo usando la función \textbf{rand\_r}. Para cada valor de la semilla 
se genera un valor pseudoaleatorio al cual se lo fuerza a caer en el intervalo pasado por parametro, tomandole módulo la distancia entre el máximo y 
el mínimo, y luego sumandole el mínimo. Una vez calculado el tiempo, se hace la llamada al uso del dispositivo de I/O.

Ejemplo: 

      \begin{figure}[H]
        \includegraphics[scale=0.5]{ejercicio1}
      \end{figure}

El gráfico esta corriendo un lote de 4 tareas de tipo \textbf{TaskConsola}. El algoritmo de $scheduling$ utilizado para representar la interacción 
entre las tareas es \textbf{First Come, First Served}. La cantidad de llamadas bloqueantes son 2 y el intervalo de 
tiempo para cada llamada es entre 2 y 6. Podemos observar como efectivamente la duración de cada llamada bloqueante pertenece a ese 
intervalo

\newpage

\section{Ejercicio 2}

El ejercicio consiste en simular la situación que enfrenta nuestro querido amigo Rolando, el cual quiere correr un algoritmo a la vez que escucha 
música y consume drogas. El algoritmo que corre hace un uso intensivo del cpu por 100 ciclos, mientras que la música y el internet realizan una 
cantidad determinada de llamadas bloqueantes. La música realiza 20 y el internet 25, cada una de duración variable entre 2 y 4 ciclos. La manera 
de generar la duración pseudoaleatoria de los ciclos de las llamadas bloqueantes fue la misma que la utilizada en el ejercicio previo, a través 
de la funcion \textbf{generate}. El algoritmo de $scheduling$ utilizado para este ejercicio fue \textbf{First Come, First Served}.

Ejemplos:

  






\end{document}

